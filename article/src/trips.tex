\section{Data Preparation/Trip}\label{sec:trips}

The data set is split into trips by annotating each record with a trip identifier, \tid.
A trip is defined as a collection of consecutive records with the same vehicle identifier, \vid.
Idle time, i.e. when the engine is running but the vehicle is not moving, is persumed to be an important factor for fuel consumption. 
We therefore need to ensure that records recorded when the vehicle is idling is included in the trips. 

A trip is hence defined from when the engine of the vehicle is turned on.
In order not to split a trip into two just because the engine stalls, we say that a trips ends when two consecutive records are more than some timeframe apart.
Figure~\ref{fig:TimeTrips} shows the number of trips at different timeframes from 5 seconds between two trips to 200 seconds.
We see that the curve flattens around 20 and we choose this as the timeframe. 
\begin{figure}[htb]
\centering
\includegraphics[width=0.5\textwidth]{../src/images/TimeTrips.png}
\caption{Number of trips at different timeframes}
\label{fig:TimeTrips}
\end{figure}

Too short trips with too few records will not give a usable image of which factors influence fuel consumption.
Figure~\ref{fig:LengthTrips} show the number of trips with different minimum limits on number of records in a trip.
\begin{figure}[htb]
\centering
\includegraphics[width=0.5\textwidth]{../src/images/TripsLength.png}
\caption{Number of trips at different lengths}
\label{fig:LengthTrips}
\end{figure}