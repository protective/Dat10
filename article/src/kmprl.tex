\section{Kilometers per liter fuel}

How many kilometers(km) per liter fuel (km/l) a vehicle drives will indicate how fuel efficient the vehicle is.
The main goal of a vehicle is transportation and hence, the more kilometers one can drive per liter fuel, the cheaper it is.

Let $\kml{t}$ be the total number of km driven in trip $t$ devided by the total fuel consumption of trip $t$, and let $\kml{v}$ be the average of $\kml{t}$ for all trips that vehicle $v$ has driven.

Figure~\ref{fig:kmlTrips} plots all $\kml{t}$ with trips on the x-axis, $\kml{t}$ on the y-axis and a plot symbol for each vehicle.
We see that the majority of the trips have a $\kml{t}$ between 5 and 10 km/l, but also that many trips have a near zero value.
The latter is because they do not drive anywhere.
This is very consistent with what can be expected from minibusses.

All trips are devided into classes based on their $\kml{t}$, such that we later can classify the different factors. %TODO: Uklart
All trips with $\kml{t}\leq4$ are in the \fuelLow class. These trips have an distinctively low km/l and lie out side of the normal range. 
The remaining trips are grouped in two to classes, \fuelMedium and \fuelHigh in order to distinquish the best from the rest.
\fuelMedium contains trips with $4 < \kml{t} < 8$ and \fuelHigh contains trips with $\kml{t}\geq8$.
Had we only used two classes, e.g. 0 to 4 and above 4, will allow us to identify very bad trips but not average versus good. 
One could also have a class containing trips with more than 10 km/l but only 94 trips will be in this class. %CHECK
\begin{figure}[htb]
\centering
\includegraphics[width=0.5\textwidth]{../src/images/kmlTrips.png}
\caption{km/l for all trips}
\label{fig:kmlTrips}
\end{figure}

