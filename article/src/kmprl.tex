\section{Classifying Data}
%TODO: Why those classes?

Fuel consumption is a measure of effectiveness, and how many kilometers per liter fuel (km/l) a vehicle drives will indicate how fuel efficient the vehicle is.
The main goal of a vehicle is transportation and hence, the more kilometers one can drive per liter fuel, the cheaper it is.




Let $\kml{\trip{}}$ be the total number of km driven devided by the total fuel consumption of trip $\trip{}$ and let $kml_{vid}$ be the average of $\kml{\trip{}}$ for all trips that vehicle \vid{\trip{}} has driven.

Figure~\ref{fig:kmlTrips} plots $\kml{\trip{i}}$ for all trips ordered by time and grouped by vehicles.
All trips are classified into three classes based on their $\kml{\trip{i}}$.
These are marked as horisontal lines on Figure~\ref{fig:kmlTrips}.
We need to be able to distinguish fuel efficient trips from fuel inefficient trips.  
We see that the majority of the trips have a $\kml{\trip{i}}$ between 5 and 12 km/l, but that many trips have a lesser value.
$\kml{\trip{i}}=0$ when the trip is 0 km long.
Overall, the values are very consistent with what can be expected from minibusses.
Table~\ref{tb:tripStatistics} lists $\kml{\vid{}}$ and the standard sample deviation of the four vehicles and on the total data set.

\begin{table}
\begin{center}
\begin{tabular}{|c|c|c|}\hline %CHECK
Vehicle id & Average $km/l$ & Standard Deviation \\\hline
67 & 6.78 & 2.69\\\hline
58 & 5.82 & 2.55\\\hline
40 & 7.38 & 2.77\\\hline
10 & 6.75 & 2.66\\\hline\hline
Average & 6.73 & 2.72\\\hline
\end{tabular}
\end{center}
\caption{Vehicle statistics}\label{tb:tripStatistics}
\end{table}

The class \fuelLow is made of the trips with unusually low $\kml{\trip{}}$, being those where km/l is less than the average km/l minus the standard sample deviation.
\[\kml{\trip{i}} < c_l\]
\[c_l = \frac{\sum_{\trip{i}\in \mathbb{T}}\kml{\trip{i}}}{\mid \mathbb{T}\mid} - \sigma_{\trip{i}\in \mathbb{T}}(\kml{\trip{i}}) = 4.02\]%CHECK

where $\sigma_{\trip{i}\in \mathbb{T}}(\kml{\trip{i}})$ is the standard sample deviation.
The remaining trips are grouped into two classes, \fuelMedium and \fuelHigh.
We split the main group into two in order to distinqush the more fuel efficient trips from the less fuel efficient.
More classes will not be more beneficial and less will prevent detailed analysis. %TODO: Not good
The classes \fuelMedium and \fuelHigh is split by the average \kml{\trip{i}} of the remaning trips.
\[c_h = \frac{\sum_{\trip{i}\in \mathbb{T'}}\kml{\trip{i}}}{\mid \mathbb{T'}\mid} = 7.7\]%CHECK
\[\mathbb{T}' = \{\trip{j}\in \mathbb{T}\mid \kml{\trip{j}} > c_l\}\]
Hence class \fuelMedium contains trips with
\[c_l \leq \kml{\trip{}} < c_h\]
and class \fuelHigh contains trips with
\[c_h \leq \kml{\trip{}}\]
 

Some odd values can also be seen around $2.5$ and $3.33 km/l$ where several trips have the exact same $\kml{\trip{}}$.
This is due to inaccuracies in the measurements.

\begin{figure}[htb]
\centering
\includegraphics[width=0.45\textwidth]{../src/e_images/kmlTrips.png}
\caption{km/l for all trips}
\label{fig:kmlTrips}
\end{figure}

