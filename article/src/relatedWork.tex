\section{Related Work}

The authors of \cite{EcodriveEnergy} investigates the impact of providing real time eco-driving advices to the drivers based on real-time traffic speed, density and flow. 
They find that a reduction in fuel consumption of 10-20\% can be achieved without significant imcrease in travel time, and that the effect is greater in severly congested senarios. 
The authors of \cite{EvalEcoDriving} also investigate how real time feedback affect driving behaviour. 
From simply displaying the instantanious fuel economy to 20 sample drivers they show a reduction of 6 \% in fuel consumption on city streets and 1\% on highways. 
Most of the drivers where willing to adopt eco-driving advises after the study.
The long-term effects of eco-driving courses are evaluated in \cite{Beusen}.
A study on 10 vehicles over 10 months shows a mean reduction in fuel consumption of 5.8 \%, but that the effect was very different from individual to individual. 20\% saw no fuel reduction. 
The thesis \cite{TruckDriver} investigates why the bennefit of eco-driving decrease over time.
The study finds that group behaviour needs to be taking into account when teaching eco-driving principles. 

The exact eco-driving advices vary from reference to reference.
Both \cite{EcodrivingAdvice} and \cite{KorGront} details a number of advises.
%TODO : Elaborate 

Eco-routing is about saving fuel by finding the most fuel efficient routes. 
GreenGPS \cite{GreenGPS} is an example of participatory approach to eco-routing and they see a 10 \% reduction in fuel consumption.
EcoMark \cite{EcoMark} is an evaluation framework for evaluating evironmental models. 
Eleven known models for environmental impact are evaluated to investigate whether they can be used to do eco-driving and eco-routing.
The evaluation finds that instatanious models can be used for eco-driving and aggregated models can be used for eco-routing.
The INTEGRATION model framework \cite{IMF} is a model for quantifying environmental impact on a microscopic level.
The study showed that the predicted the emissions and fuel consumption are consistent with actual data from a field study when the vehicles do not accelerate. %TODO: What do they find. %CHECK

%TODO: We have a cite on the accuracy of GPS data if we want it.