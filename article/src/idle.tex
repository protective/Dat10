\section{Avoid Idling}

Avoiding idling or minimising idle time is a factor in eco-driving as fuel is still consumed when the engine is running even though the vehicle is not moving.
The driver is hence consuming unneccesary fuel when idling.
A vehicle is idling when the engines rounds per minute (RPM) is above zero while the speed is zero for at least 2 recordings.
A vehicle is hence idling when for example waiting at a red light or in a queue. 
Figure~\ref{fig:idleRange2} shows how often the vehiceles idle and how long.
The idle time is split into ranges of ten seconds and displaied on the x-axis, e.g. 10.0 indicates the time range 10-19 seconds. 
The number of times each vehicle idles in the given time range is displaied on the y-axis in a logarithmic scale. 
It is clear to see that most of the vehicles idles less that 110 seconds which correlates with the longes circulation times of traffic ligths. 
\begin{figure}[htb]
\centering
\includegraphics[width=0.5\textwidth]{../src/images/idleRange2.png}
\caption{Minimum idle duration}
\label{fig:idleRange2}
\end{figure}

Figure~\ref{fig:idleRange3} shows how much fuel has been consumed in the idling periodes of more than 110 seconds. 
Notice that only four of the vehicles does this.
Fuel consumption is shown on the y-axis and the number of seconds in the idle period is shown on the x-axis with a logaritmic scale.
We see that very little fuel is consumed in these idling periods even for the vehicle that idles for several hours.

\begin{figure}[htb]
\centering
\includegraphics[width=0.5\textwidth]{../src/images/idleRange3.png}
\caption{Minimum idle duration}
\label{fig:idleRange3}
\end{figure}



A vehicle is idling when the engines rounds per minute (RPM) is above zero while the speed is zero for at least 50 consecutive recordings. %CHECK
Let $\vrpm$ and $\vspeed$ be the RPM and speed of vehicle $v$ at time $t$, then a vehicle is ideling when
\[\mbox{\textbf{for }} i, \dots ,t, \dots, i+50 \mbox{: }\vrpm> 0 \wedge \vspeed=0 \]
A vehicle must be in idling at least $50$ seconds in order to eliminate involuntary idle times at for example traffic lights or due to small disturbances is the traffic flow.
Figure~\ref{fig:idleDuration} plots a graph of the number of records in which a vehicle is idling at different minimum durations. 

\begin{figure}[htb] %TODO: Not updated
\centering
\includegraphics[width=0.5\textwidth]{../src/images/idleDuration.png}
\caption{Minimum idle duration}
\label{fig:idleDuration}
\end{figure}

Figure~\ref{fig:idleTime} plots the km/l for all trips as a function of how many seconds the vehicle has been idling.
The size of the plot indicates the amount of fuel used in the trip and the color signals what class it has been asigned to.
TODO: analyse
\begin{figure}[htb]
\centering
\includegraphics[width=0.5\textwidth]{../src/images/idleTime.png}
\caption{Idle time}
\label{fig:idleTime}
\end{figure}

It might give a more fair result, if the numbers from Figure~\ref{fig:idleTime} are normalised such that the x-axis plots how many percent of the records are an idle state compared to the number of records in the trip.
Figure~\ref{fig:idlePercent} plots the km/l for all trips as a function of how many percent of the trip the vehicle has been idling.
The size of the plot indicates the amount of fuel used in the trip and the color signals what class it has been asigned to.
TODO: analyse
\begin{figure}[htb]
\centering
\includegraphics[width=0.5\textwidth]{../src/images/idlePercent.png}
\caption{Idle percent}
\label{fig:idlePercent}
\end{figure}

Figure~\ref{fig:idleClassPercent} abstracts away the actual km/l and only plots the class distibution as a function of he idle percent. 
The purple line indicates how many trips have this idle percentage. 
The graph is cut off at 60 \% because of too little data.
It is evident that the majority of the trips in the \fuelHigh class avoids idling. 
TODO: analyse

\begin{figure}[htb]
\centering
\includegraphics[width=0.5\textwidth]{../src/images/idle2.png}
\caption{Class distribution }
\label{fig:idleClassPercent}
\end{figure}

\begin{figure}[htb]
\centering
\includegraphics[width=0.5\textwidth]{../src/images/idle3.png}
\caption{Class distribution}
\label{fig:idleClassTime}
\end{figure}

