\section{Avoid Idling}\label{sec:idle}

%TODO: Make sure the vehicles have the same color and order on all pictures.  

Avoiding idling or minimising idle time is a factor in eco-driving as fuel is still consumed when the engine is running even though the vehicle is not moving.
The driver is hence consuming unneccesary fuel when idling.

We say a vehicle is \textit{stopped} iff. the RPM i above zero and the speed is zero for at least 2 consecutive recordings.
\begin{align}
\stopp{\rec{i+1}} = true &\mbox{ iff. } \nonumber\\
&\rpm{\rec{i}} > 0 \wedge \rpm{\rec{i+1}} > 0 \wedge \nonumber\\
&\speed{\rec{i}} = 0 \wedge \speed{\rec{i+1}} = 0 \nonumber
\end{align}

A vehicle is hence stopped when for example waiting at a red light, in a queue or parked with the engine on. 
A \textit{stopped period} is a sequence of records where $\stopp{\rec{}}$ is true as per Section~\ref{sec:periods}.
But when looking at idling, we are not interesseted in all stopped periods.
All stops near traffic lights is not idling and very short stops are neigher.  
Figure~\ref{fig:idleRange2} shows how often the vehiceles are stopped outside traffic lights and for how long. 
The periods are combined into ranges of 100 seconds and displayed on the x-axis, e.g. 100.0 indicates the time range 100-199 seconds. 
The number of times each vehicle is stopped in the given time range is displayed on the y-axis in a logarithmic scale.
Note that some periods have been ommited in this graph due to too few occurences.
It is clear to see that most of the vehicles stop for less 100 seconds which correlates with the longest circulation times of traffic ligths. 

\begin{figure}[htb]
\centering
\includegraphics[width=0.45\textwidth]{../src/e_images/idleRange2.png}
\caption{Number of idle periods at different ranges}
\label{fig:idleRange2}
\end{figure}

Figure~\ref{fig:idleDuration} shows the total number of stopped periods with different minimum durations of the periods.
The curve flattens around 250 $s$, a little over $4$ minutes.
An \textit{idle period} is therefore a stopped period that longer than 250 $s$ and not near a traffic light.
Involuntary stopped periods such as queues and alike are thereby eliminated.
\begin{figure}[htb]
\centering
\includegraphics[width=0.45\textwidth]{../src/e_images/idleDuration.png}
\caption{Minimum idle duration}
\label{fig:idleDuration}
\end{figure}

Figure~\ref{fig:idleRange3} shows how much fuel is consumed in the idle periods. 
Fuel consumption is shown on the y-axis and the number of seconds of the idle period is shown on the x-axis.
A linear regression line has been plottet for each vehicle
We see that few idles for more than 1300 seconds ($\sim$ 21 minuts) but that these use between 0.5 and 2 liters of fuel each time.
Following the regression lines it can be estimated that between 0.85 and 1.25 liters of fuel are used per idling hour.  
\begin{figure}[htb]
\centering
\includegraphics[width=0.45\textwidth]{../src/e_images/idleRange3.png}
\caption{Fuel comsumption when idling}
\label{fig:idleRange3}
\end{figure}

Table~\ref{tb:idleFuel} shows how much each vehicle idles and how much fuel is used.
Vehicle 58 idles the most and consumes about a quarter of a liter of fuel each day on idling. 
Vehicle 67 only idles around 4 minuts a day and is the vehicle that consumes least fuel on idling.
Vehicle 10 and 40 use almost the same amount of fuel, but vehicle 40 idles much less than vehicle 10. 
This makes vehicle 40 more fuel efficient when idling. 
\begin{table}
\centering
\begin{tabular}{|c|c|c|c|c|}\hline%CHECK
			& \multicolumn{2}{c|}{Idle} & \multicolumn{2}{c|}{Fuel}\\
Vehicle id & $s$ & $min/day$ & $l$& $l/day$\\\hline
        40 & 109,138 & 7.86 & 31.17 & 0.13\\\hline
        58 & 179,118 & 12.64 & 60.11 & 0.25\\\hline
        67 &  67,312 & 4.75 & 18.37 & 0.08\\\hline
        10 & 153,083 & 10.90 & 35.56 & 0.15\\\hline
%Average & 36.30 & 127.162 s $\sim$ ?? hours & 1.03\\\hline %TODO: Do average
\end{tabular}
\caption{Time and fuel consumption in idle}\label{tb:idleFuel}
\end{table}
%select i.vehicleid, idle, (idle/60)/(time/60/60/24) as idleday, fuel, fuel/(time/60/60/24) as fuelday from (select vehicleid, sum(stopped)::float as idle, sum(fuel)::float as fuel from e_idledatatl where stopped > 250 group by vehicleid)i, (select vehicleid, EXTRACT(EPOCH FROM max(timestamp)-min(timestamp))::float as time from e_gps_can_data group by vehicleid)t where i.vehicleid=t.vehicleid;

%select i.vehicleid, idle, time, (idle/60)/(time/60/60/24) as idleday, fuel, fuel/(time/60/60/24) as fuelday from (select vehicleid, sum(stopped)::float as idle, sum(fuel)::float as fuel from e_idledatatl where stopped > 250 group by vehicleid)i, (select vehicleid, EXTRACT(EPOCH FROM max(timestamp)-min(timestamp))::float as time from e_gps_can_data group by vehicleid)t where i.vehicleid=t.vehicleid;

It will also be interesting to investigate whether how much of a trip is in an idle state has an infulence on the trips km/l.
Figure~\ref{fig:idleClassPercent} shows the class distribution of all trips grouped by how much of the trip is in an idle period.
The yellow line indicates the number of trips on a logaritmic scale.
We clearly see that of the trips with a small idle percentage most are in the class \fuelHigh and vice versa.
This strongly suggests that there is a correlation between idle and fuel efficiency.
Figure~\ref{fig:idleClassTime} shows the class distribution over the length of the pure idle periods split into ranges.
We do not see the same clear distinction in this plot as for Figure~\ref{fig:idleClassPercent}.
This might indicate that the trips from class \fuelLow do not necesarily idle for long periods but idles more often. 
\begin{figure}[htb]
\centering
\includegraphics[width=0.45\textwidth]{../src/e_images/idle2.png}
\caption{Class distribution of percentage in pure idle periods}
\label{fig:idleClassPercent}
\end{figure}
\begin{figure}[htb]
\centering
\includegraphics[width=0.45\textwidth]{../src/e_images/idle3.png}
\caption{Class distribution of time in pure idle periods}
\label{fig:idleClassTime}
\end{figure}

