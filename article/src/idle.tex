\section{Avoid Idling}

Avoiding idling or minimising idle time is a factor in eco-driving as fuel is still consumed when the engine is running even though the vehicle is not moving.
The driver is hence consuming unneccesary fuel when idling.
A vehicle is idling when the engines rounds per minute (RPM) is above zero while the speed is zero for at least 2 recordings.
A vehicle is hence idling when for example waiting at a red light or in a queue or when parked with the engine running. 
Let $\vrpm$ and $\vspeed$ be the RPM and speed of vehicle $v$ at timestamp $t$ and let $\vidle$ be true if the vehicle is ideling at timestamp $t$.
Then,
\[\vidle = true \mbox{\textbf{ iff. }} \vrpm> 0 \wedge \vspeed=0 \]
An \textit{idle period} is a set of consecutive records where the vehicle is ideling,
\[\vidlep = \{r \mid v_{r.t}^{idle}= true \}\]
and $\timelength{\vidlep}$ is how many seconds the idle period spans.

Figure~\ref{fig:idleRange2} shows how often the vehiceles idle and for how long.
The idle periods are split into ranges of 100 seconds and displaied on the x-axis, e.g. 100.0 indicates the range 100-199 seconds. 
The number of times each vehicle idles in the given time range is displaied on the y-axis in a logarithmic scale. 
All ranges plotted on the x-axis has at least one occurrence.
It is clear to see that most of the vehicles idles less that 100 seconds which correlates with the longest circulation times of traffic ligths. 
\begin{figure}[htb]
\centering
\includegraphics[width=0.5\textwidth]{../src/canbusImages/idleRange2.png}
\caption{Number of idle periods in ranges}
\label{fig:idleRange2}
\end{figure}

Figure~\ref{fig:idleRange3} shows how much fuel has been consumed in the idling periods of more than 100 seconds. 
Fuel consumption is shown on the y-axis and the number of seconds in the idle period is shown on the x-axis.
We see that little fuel is consumed in these idling periods even for the vehicle that idles for more than an hour.
\begin{figure}[htb]
\centering
\includegraphics[width=0.5\textwidth]{../src/canbusImages/idleRange3.png}
\caption{Fuel consumption in idle periods}
\label{fig:idleRange3}
\end{figure}

Table~\ref{tb:idleFuelConsumption} list the fuel consumption of the four vehicles while idling.
Vehicles 67 and 58 idles significantly less than vehicles 40 and 10 and use around 30 liters less.
On average, they use about 1.1 $l/h$, except vehicle 10 how use a little under 1 $l/h$.
The variances might be due to small differences in the construction of the vehicles and measuring inaccuracies.
\begin{table}
\centering
\begin{tabular}{l|r|r|r}
Vehicle id & Time ($s$) & Fuel ($l$) &  Fuel ($l/h$)\\\hline
67 & 278,119 & 98.175 & 1.275\\\hline
58 & 273,872 & 98.256 & 1.293\\\hline
40 & 401,384 & 127.200 & 1.146\\\hline
10 & 490,721 & 128.754 & 0.947
\end{tabular}
\caption{Fuel consumption while idling}
\label{tb:idleFuelConsumption}
\end{table}


\begin{comment}
A vehicle is idling when the engines rounds per minute (RPM) is above zero while the speed is zero for at least 50 consecutive recordings. %CHECK
Let $\vrpm$ and $\vspeed$ be the RPM and speed of vehicle $v$ at time $t$, then a vehicle is ideling when
\[\mbox{\textbf{for }} i, \dots ,t, \dots, i+50 \mbox{: }\vrpm> 0 \wedge \vspeed=0 \]
A vehicle must be in idling at least $50$ seconds in order to eliminate involuntary idle times at for example traffic lights or due to small disturbances is the traffic flow.
Figure~\ref{fig:idleDuration} plots a graph of the number of records in which a vehicle is idling at different minimum durations. 

\begin{figure}[htb] %TODO: Not updated
\centering
\includegraphics[width=0.5\textwidth]{../src/images/idleDuration.png}
\caption{Minimum idle duration}
\label{fig:idleDuration}
\end{figure}

Figure~\ref{fig:idleTime} plots the km/l for all trips as a function of how many seconds the vehicle has been idling.
The size of the plot indicates the amount of fuel used in the trip and the color signals what class it has been asigned to.
TODO: analyse
\begin{figure}[htb]
\centering
\includegraphics[width=0.5\textwidth]{../src/images/idleTime.png}
\caption{Idle time}
\label{fig:idleTime}
\end{figure}

It might give a more fair result, if the numbers from Figure~\ref{fig:idleTime} are normalised such that the x-axis plots how many percent of the records are an idle state compared to the number of records in the trip.
Figure~\ref{fig:idlePercent} plots the km/l for all trips as a function of how many percent of the trip the vehicle has been idling.
The size of the plot indicates the amount of fuel used in the trip and the color signals what class it has been asigned to.
TODO: analyse
\begin{figure}[htb]
\centering
\includegraphics[width=0.5\textwidth]{../src/images/idlePercent.png}
\caption{Idle percent}
\label{fig:idlePercent}
\end{figure}

Figure~\ref{fig:idleClassPercent} abstracts away the actual km/l and only plots the class distibution as a function of he idle percent. 
The purple line indicates how many trips have this idle percentage. 
The graph is cut off at 60 \% because of too little data.
It is evident that the majority of the trips in the \fuelHigh class avoids idling. 
TODO: analyse

\begin{figure}[htb]
\centering
\includegraphics[width=0.5\textwidth]{../src/images/idle2.png}
\caption{Class distribution }
\label{fig:idleClassPercent}
\end{figure}

\begin{figure}[htb]
\centering
\includegraphics[width=0.5\textwidth]{../src/images/idle3.png}
\caption{Class distribution}
\label{fig:idleClassTime}
\end{figure}
\end{comment}
