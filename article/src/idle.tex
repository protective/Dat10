\section{Avoid Idling}

Avoiding idling or minimising idle time is a factor in eco-driving as fuel is still consumed when the engine is running even though the vehicle is not moving.
The driver is hence consuming unneccesary fuel when idling.
A vehicle is idling when the engines rounds per minute (RPM) is above zero while the speed is zero for at least 2 recordings.
A vehicle is hence idling when for example waiting at a red light or in a queue or when parked with the engine running. 
Let $\vrpm$ and $\vspeed$ be the RPM and speed of vehicle $v$ at timestamp $t$ and let $\vidle$ be true if the vehicle is idling at timestamp $t$.
Then,
\[\vidle = true \mbox{\textbf{ iff. }} \vrpm> 0 \wedge \vspeed=0 \]
An \textit{idle period} is a set of consecutive records where the vehicle is idling,
\[\vidlep = \{r \mid v_{r.t}^{idle}= true \}\]
and $\timelength{\vidlep}$ is duration of the idle period in seconds.

When analysing idle, we use the un-map-matched data, as many idling records are removed during this process. 

Figure~\ref{fig:idleRange2} shows how often the vehiceles idle and for how long.
The idle periods are split into ranges of 100 seconds and displayed on the x-axis, e.g. 100.0 indicates the range 100-199 seconds. 
The number of times each vehicle idles in the given time range is displayed on the y-axis in a logarithmic scale. 
All ranges plotted on the x-axis has at least one occurrence.
It is clear to see that most of the vehicles idles less that 100 seconds which correlates with the longest circulation times of traffic ligths. 
\begin{figure}[htb]
\centering
\includegraphics[width=0.5\textwidth]{../src/d_images/idleRange2.png}
\caption{Number of idle periods in ranges}
\label{fig:idleRange2}
\end{figure}

Figure~\ref{fig:idleRange3} shows how much fuel has been consumed in the idling periods of more than 100 seconds. 
Fuel consumption is shown on the y-axis and the number of seconds in the idle period is shown on the x-axis.
We see that little fuel is consumed in these idling periods even for the vehicle that idles for more than an hour.
\begin{figure}[htb]
\centering
\includegraphics[width=0.5\textwidth]{../src/d_images/idleRange3.png}
\caption{Fuel consumption in idle periods}
\label{fig:idleRange3}
\end{figure}

Table~\ref{tb:idleFuelConsumption} list the fuel consumption of the four vehicles while idling.
Vehicles 67 and 58 idles significantly less than vehicles 40 and 10 and use around 30 liters less.
On average, they use about 1.1 $l/h$, except vehicle 10 how use a little under 1 $l/h$.
The variances might be due to small differences in the construction of the vehicles and measuring inaccuracies.
\begin{table}
\centering
\begin{tabular}{l|r|r|r}
Vehicle id & Time ($s$) & Fuel ($l$) &  Fuel ($l/h$)\\\hline
67 & 278,119 & 98.175 & 1.275\\\hline
58 & 273,872 & 98.256 & 1.293\\\hline
40 & 401,384 & 127.200 & 1.146\\\hline
10 & 490,721 & 128.754 & 0.947
\end{tabular}
\caption{Fuel consumption while idling}
\label{tb:idleFuelConsumption}
\end{table}

It will also be interesting to see how much of a trip is an idle period and if this percentage has an influence on the vehicles fuel efficiency.
Let trips be defined as in Section~\ref{sec:trips}.
First we need to investigate whether a lower limit on the duration of the idle periods is necessary.
Figure~\ref{fig:idleDuration} plots the number of 

Lower limit on idle period
\begin{figure}[htb]%TODO: not updated
\centering
\includegraphics[width=0.5\textwidth]{../src/d_images/idleDuration.png}
\caption{Minimum idle period duration}
\label{fig:idleDuration}
\end{figure}

Percent of trip is idle.
\begin{figure}[htb]%TODO: not updated
\centering
\includegraphics[width=0.5\textwidth]{../src/d_images/idle2.png}
\caption{Idle percent}
\label{fig:idlePercent}
\end{figure}

Idle period duration 
\begin{figure}[htb]%TODO: not updated
\centering
\includegraphics[width=0.5\textwidth]{../src/d_images/idle3.png}
\caption{Idle time}
\label{fig:idleTime}
\end{figure}
