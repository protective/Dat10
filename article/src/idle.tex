\section{Avoid Idling}

Avoiding idling or minimising idle time is a factor in eco-driving as fuel is still consumed when the engine is running even though the vehicle is not moving.
The driver is hence consuming unneccesary fuel when idling.
A vehicle is idling when the engines rounds per minute (RPM) is above zero while the speed is zero for at least 2 consecutive recordings.
A vehicle is hence also idling when for example waiting at a red light or in a queue. 

Let 


Figure~\ref{fig:idleRange2} shows how often the vehiceles idle and how long.
The idle time is split into ranges of ten seconds and displaied on the x-axis, e.g. 100.0 indicates the time range 100-199 seconds. 
The number of times each vehicle idles in the given time range is displaied on the y-axis in a logarithmic scale. 
It is clear to see that most of the vehicles idles less that 100 seconds which correlates with the longest circulation times of traffic ligths. 
\begin{figure}[htb]
\centering
\includegraphics[width=0.5\textwidth]{../src/d_images/idleRange2.png}
\caption{Minimum idle duration}
\label{fig:idleRange2}
\end{figure}

Figure~\ref{fig:idleRange3} shows how much fuel has been consumed in the idling periodes of more than 100 seconds. 
Fuel consumption is shown on the y-axis and the number of seconds in the idle period is shown on the x-axis.
We see that few idles for more than 25 minutes and that even these do not use more than 2 liters of fuel.
\begin{figure}[htb]
\centering
\includegraphics[width=0.5\textwidth]{../src/d_images/idleRange3.png}
\caption{Minimum idle duration}
\label{fig:idleRange3}
\end{figure}

It will also be interessting to investigate whether how much of a trip is in an idle state has an infulence on the trips km/l.
Let $\vrpm$ and $\vspeed$ be the RPM and speed of vehicle $v$ at time $t$
A vehicle has a \textit{pure idle period} when


\[\]


\[\mbox{\textbf{for }} i, \dots ,t, \dots, i+50 \mbox{: }\vrpm> 0 \wedge \vspeed=0 \]
A vehicle must be in idling at least $50$ seconds in order to eliminate involuntary idle times at for example traffic lights or due to small disturbances is the traffic flow.
Figure~\ref{fig:idleDuration} plots a graph of the number of records in which a vehicle is idling at different minimum durations. 

\begin{figure}[htb]
\centering
\includegraphics[width=0.5\textwidth]{../src/d_images/idleDuration.png}
\caption{Minimum idle duration}
\label{fig:idleDuration}
\end{figure}

\begin{comment}
Figure~\ref{fig:idleTime} plots the km/l for all trips as a function of how many seconds the vehicle has been idling.
The size of the plot indicates the amount of fuel used in the trip and the color signals what class it has been asigned to.
TODO: analyse
\begin{figure}[htb]
\centering
\includegraphics[width=0.5\textwidth]{../src/images/idleTime.png}
\caption{Idle time}
\label{fig:idleTime}
\end{figure}

It might give a more fair result, if the numbers from Figure~\ref{fig:idleTime} are normalised such that the x-axis plots how many percent of the records are an idle state compared to the number of records in the trip.
Figure~\ref{fig:idlePercent} plots the km/l for all trips as a function of how many percent of the trip the vehicle has been idling.
The size of the plot indicates the amount of fuel used in the trip and the color signals what class it has been asigned to.
TODO: analyse
\begin{figure}[htb]
\centering
\includegraphics[width=0.5\textwidth]{../src/images/idlePercent.png}
\caption{Idle percent}
\label{fig:idlePercent}
\end{figure}

Figure~\ref{fig:idleClassPercent} abstracts away the actual km/l and only plots the class distibution as a function of he idle percent. 
The purple line indicates how many trips have this idle percentage. 
The graph is cut off at 60 \% because of too little data.
It is evident that the majority of the trips in the \fuelHigh class avoids idling. 
TODO: analyse

\begin{figure}[htb]
\centering
\includegraphics[width=0.5\textwidth]{../src/images/idle2.png}
\caption{Class distribution }
\label{fig:idleClassPercent}
\end{figure}

\begin{figure}[htb]
\centering
\includegraphics[width=0.5\textwidth]{../src/images/idle3.png}
\caption{Class distribution}
\label{fig:idleClassTime}
\end{figure}

\end{comment}

