\section{Data Description}
A number of data sources can be used to gather data.
GPS data can easily be collected from any vehicle equiped with a GPS system, through which for example a timestamp, longitude, latitude and speed are available.
CANBus data provides information about the state of the vehicle.  %TODO: Reformulate
Retriving CANBus data is not as easy as GPS data, and little data is therefore available. 
CANBus data is for example the engines rounds per minute, fuel consumption and temperature. 
The hauliers may also provide additional information such as vehicle model, capacity and weight. 

Table~\ref{tb:dataDescription} lists the data columns of the provided data set.
GPS and CANBus data is colleted for six vehicles over a timeperiod.
Most of the data is recorded with 1 hertz frequency, but some data is collected with a larger frequency.
The data set contains 6,365,359 records with about 1 mio. records from five vehicles and half a milion from one vehicle.
All vehicles are assumed to be compareable.%TODO
Of the four columns for fuel comsumption, \var{totalconsumed} is the most accurate measure.
The instantaneous values, i.e. \var{actualconsumed} and \var{actual\_km\_l}, is said to be undependable by the data provider and should not be used. 
\var{fuellevel} is the fuel level in the tank in percentage (100 \% is full and 0 \% is empty) and \var{totalconsumed} is the amount of fuel consumed.
\var{totalconsumed} only has a granularity of half a liter, which means it can only be used at an agreegated level of at least a trip in order to get usable results.

Non of the data provided by the haulier is available.
